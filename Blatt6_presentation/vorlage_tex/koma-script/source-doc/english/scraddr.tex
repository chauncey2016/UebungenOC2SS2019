% ======================================================================
% scraddr.tex
% Copyright (c) Markus Kohm, 2001-2018
%
% This file is part of the LaTeX2e KOMA-Script bundle.
%
% This work may be distributed and/or modified under the conditions of
% the LaTeX Project Public License, version 1.3c of the license.
% The latest version of this license is in
%   http://www.latex-project.org/lppl.txt
% and version 1.3c or later is part of all distributions of LaTeX 
% version 2005/12/01 or later and of this work.
%
% This work has the LPPL maintenance status "author-maintained".
%
% The Current Maintainer and author of this work is Markus Kohm.
%
% This work consists of all files listed in manifest.txt.
% ----------------------------------------------------------------------
% scraddr.tex
% Copyright (c) Markus Kohm, 2001-2018
%
% Dieses Werk darf nach den Bedingungen der LaTeX Project Public Lizenz,
% Version 1.3c, verteilt und/oder veraendert werden.
% Die neuste Version dieser Lizenz ist
%   http://www.latex-project.org/lppl.txt
% und Version 1.3c ist Teil aller Verteilungen von LaTeX
% Version 2005/12/01 oder spaeter und dieses Werks.
%
% Dieses Werk hat den LPPL-Verwaltungs-Status "author-maintained"
% (allein durch den Autor verwaltet).
%
% Der Aktuelle Verwalter und Autor dieses Werkes ist Markus Kohm.
% 
% Dieses Werk besteht aus den in manifest.txt aufgefuehrten Dateien.
% ======================================================================
%
% Chapter about scraddr of the KOMA-Script guide
% Maintained by Jens-Uwe Morawski (with help from Markus Kohm)
%
% ----------------------------------------------------------------------
%
% Kapitel ueber scraddr in der KOMA-Script-Anleitung
% Verwaltet von Jens-Uwe Morawski (mit Unterstuetzung von Markus Kohm)
%
% ======================================================================

\KOMAProvidesFile{scraddr.tex}
                 [$Date: 2018-03-27 11:46:14 +0200 (Tue, 27 Mar 2018) $
                  KOMA-Script guide (chapter: scraddr)]
\translator{Jens-Uwe Morawski\and Gernot Hassenpflug\and Markus Kohm\and Karl
  Hagen}

% Date of the translated German file: 2018-02-05

\chapter{Accessing Address Files with \Package{scraddr}}%
\labelbase{scraddr}%
\BeginIndexGroup
\BeginIndex{Package}{scraddr}

The \Package{scraddr} package is a small extension to \KOMAScript{}'s letter
class and letter package. Its goal is to make access to the data in address
files easier and more flexible.

\section{Overview}\seclabel{overview}
Basically, the package provides a new loading mechanism for address files
consisting of \DescRef{\LabelBase.cmd.adrentry} and the newer
\DescRef{\LabelBase.cmd.addrentry} format entries, as described in
\autoref{cha:scrlttr2} starting on \DescPageRef{scrlttr2.cmd.adrentry}.

\begin{Declaration}
\Macro{InputAddressFile}\Parameter{file name}
\end{Declaration}%
The \Macro{InputAddressFile} command is the main command of \Package{scraddr}.
It reads the content of the address file\Index{address>file} given as its
parameter. If the file is not found, an error message is issued.

For each entry in this address file, the command generates a set of
macros to access the data. For large address files, this will require
a lot of \TeX{} memory.
%
\EndIndexGroup

\begin{Declaration}%
  \Macro{adrentry}\Parameter{Lastname}\Parameter{Firstname}%
  \Parameter{Address}\Parameter{Phone}\Parameter{F1}\Parameter{F2}%
  \Parameter{Comment}\Parameter{Key}%
  %
  \Macro{addrentry}\Parameter{Lastname}\Parameter{Firstname}%
  \Parameter{Address}\Parameter{Phone}\Parameter{F1}\Parameter{F2}%
  \Parameter{F3}\Parameter{F4}\Parameter{Key}%
  %
  \Macro{adrchar}\Parameter{initial}%
  \Macro{addrchar}\Parameter{initial}%
\end{Declaration}%
The structure of the address entries in the address file was discussed in
detail in \autoref{sec:scrlttr2.addressFile}, starting on
\DescPageRef{scrlttr2.cmd.adrentry}. The subdivision of the address file with
the help of \Macro{adrchar} or \Macro{addrchar}, also discussed there, has no
meaning for \Package{scraddr} and is simply ignored by the package.%
\EndIndexGroup

\begin{Declaration}
  \Macro{Name}\Parameter{key}%
  \Macro{FirstName}\Parameter{key}%
  \Macro{LastName}\Parameter{key}%
  \Macro{Address}\Parameter{key}%
  \Macro{Telephone}\Parameter{key}%
  \Macro{FreeI}\Parameter{key}%
  \Macro{FreeII}\Parameter{key}%
  \Macro{Comment}\Parameter{key}%
  \Macro{FreeIII}\Parameter{key}%
  \Macro{FreeIV}\Parameter{key}
\end{Declaration}%
These commands give access to data of your address file. The last parameter,
that is, parameter 8 for the \DescRef{\LabelBase.cmd.adrentry} entry and
parameter 9 for the \DescRef{\LabelBase.cmd.addrentry} entry, is the
identifier of an entry, thus the \PName{key} has to be unique and non-empty.
To guarantee safe functioning, you should use only ASCII letters in the
\PName{key}.

Furthermore, if the file contains more than one entry with the same
\PName{key} name, the last occurrence will be used.%
%
\EndIndexGroup


\section{Usage}\seclabel{usage}
\BeginIndexGroup
\BeginIndex[indexother]{Cmd}{addrentry}%
\BeginIndex[indexother]{Cmd}{adrentry}%
To use the package, we need a valid address file.  For example, the file
\File{lotr.adr} contains the following entries:
\begin{lstcode}
  \addrentry{Baggins}{Frodo}%
            {The Hill\\ Bag End/Hobbiton in the Shire}{}%
            {Bilbo Baggins}{pipe-weed}%
            {the Ring-bearer}{Bilbo's heir}{FRODO}
  \adrentry{Gamgee}{Samwise}%
            {3 Bagshot Row\\Hobbiton in the Shire}{}%
            {Rosie Cotton}{taters}%
            {the Ring-bearer's faithful servant}{SAM}
  \adrentry{Bombadil}{Tom}%
            {The Old Forest}{}%
            {Goldberry}{trill queer songs}%
            {The Master of Wood, Water and Hill}{TOM}
\end{lstcode}

The fourth parameter, the telephone number, has been left blank, since there
are no phones in Middle Earth. And as you can see, blank fields are possible.
On the other hand, you cannot simply omit an argument altogether.

\BeginIndexGroup
\BeginIndex[indexother]{Cmd}{InputAddressFile}
With the \Macro{InputAddressFile} command described above, we read the address
file into our letter document:
\begin{lstcode}
  \InputAddressFile{lotr}
\end{lstcode}
\EndIndexGroup

\BeginIndexGroup
\BeginIndex[indexother]{Cmd}{Name}%
\BeginIndex[indexother]{Cmd}{Address}%
\BeginIndex[indexother]{Cmd}{FirstName}%
\BeginIndex[indexother]{Cmd}{LastName}%
\BeginIndex[indexother]{Cmd}{FreeI}%
\BeginIndex[indexother]{Cmd}{FreeII}%
\BeginIndex[indexother]{Cmd}{FreeIII}%
\BeginIndex[indexother]{Cmd}{FreeIV}%
\BeginIndex[indexother]{Cmd}{Comment}%
With the help of the commands introduced in this chapter we can now write a
letter to old \textsc{Tom Bombadil}, in which we ask him if he can remember
two companions from olden times.
\begin{lstcode}
  \begin{letter}{\Name{TOM}\\\Address{TOM}}
     \opening{Dear \FirstName{TOM} \LastName{TOM},}
     
     Or \FreeIII{TOM}, as your beloved \FreeI{TOM} calls you. Do
     you still remember Mr \LastName{FRODO}, or more precisely
     \Name{FRODO}, since there was also Mr \FreeI{FRODO}. He was
     \Comment{FRODO} in the Third Age and \FreeIV{FRODO}. \Name{SAM},
     \Comment{SAM}, accompanied him.
      
      Their passions were very worldly. \FirstName{FRODO} enjoyed
      smoking \FreeII{FRODO}. His companion appreciated a good meal
      with \FreeII{SAM}.

      Do you remember? Certainly Mithrandir has told you much
      about their deeds and adventures.
    \closing{``O spring-time and summer-time
                and spring again after!\\
               O wind on the waterfall,
                and the leaves' laughter!''}
  \end{letter}
\end{lstcode}
You can also produce the combination of \Macro{FirstName}\Parameter{key} and
\Macro{LastName}\Parameter{key} used in the \DescRef{scrlttr2.cmd.opening} of
this letter with \Macro{Name}\PParameter{key}.

You can use the fifth and sixth parameters of the
\DescRef{\LabelBase.cmd.adrentry} or \DescRef{\LabelBase.cmd.adrentry} for any
purpose you wish. You can access them with the \Macro{FreeI} and
\Macro{FreeII} commands. In this example, the fifth parameter contains the
name of the most important person in the life of the person in the entry. The
sixth contains the person's favourite thing. The seventh parameter is a
comment or in general also a free parameter. You can access it with the
\Macro{Comment} or \Macro{FreeIII} commands. \Macro{FreeIV} is only valid for
\DescRef{\LabelBase.cmd.addrentry} entries. For
\DescRef{\LabelBase.cmd.adrentry} entries, it results in an error. You can
find more details in the next section.
%
\EndIndexGroup
\EndIndexGroup


\section{Package Warning Options}

As mentioned above, you cannot use the \Macro{FreeIV} command with
\DescRef{\LabelBase.cmd.adrentry} entries. However, you can configure how
\Package{scraddr} reacts in such a situation by package options.
Note\textnote{Attention!} that this package does not support the extended
options interface with \DescRef{maincls.cmd.KOMAoptions} and
\DescRef{maincls.cmd.KOMAoption}. You should therefore specify the options
either as global options in \DescRef{maincls.cmd.documentclass} or as local
options in \DescRef{maincls.cmd.usepackage}


\begin{Declaration}
  \Option{adrFreeIVempty}%
  \Option{adrFreeIVshow}%
  \Option{adrFreeIVwarn}%
  \Option{adrFreeIVstop}%
\end{Declaration}%
These four options let you choose from four different reactions, ranging
from \emph{ignore} to \emph{abort}, if \Macro{FreeIV} is used
within an \DescRef{\LabelBase.cmd.adrentry} entry.

\begin{labeling}[~--]{\Option{adrFreeIVempty}}
\item[\Option{adrFreeIVempty}] 
        the command \Macro{FreeIV} will be ignored
\item[\Option{adrFreeIVshow}] 
        the warning ``(entry FreeIV undefined at \PName{key})'' will be
        written in the text
\item[\Option{adrFreeIVwarn}]
        a warning is written in the logfile
\item[\Option{adrFreeIVstop}]
        the \LaTeX{} run will abort with an error message
\end{labeling}
The default setting is \Option{adrFreeIVshow}.%
\EndIndexGroup
%
\EndIndexGroup

\endinput

%%% Local Variables: 
%%% mode: latex
%%% coding: us-ascii
%%% TeX-master: "../guide"
%%% End: 
