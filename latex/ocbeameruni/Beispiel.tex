\documentclass{ocbeameruni}


\setdefaultlanguage[babelshorthands=true]{german}


% Nur benötigt für die Beispiel-Listings!
\usepackage{listings}
\lstloadlanguages{[LaTeX]TeX}
\lstset{%
  basicstyle=\small \ttfamily,
  breaklines=true,
}

\newcommand{\R}{\mathbb{R}}


\title{Organic Computing 2}
\subtitle{Lösungsvorschlag Blatt01}
\date{\today}
\author{Lukas Huhn \and Qiang Chang \and Victor Gerling \and Daniel Bossert}
\institute{%
  Universität Augsburg\\
  Institut für Informatik\\
  Lehrstuhl für Organic Computing
}


\begin{document}


\maketitle


\begin{frame}{Gliederung}
  \setbeamertemplate{section in toc}[sections numbered]
  \tableofcontents
\end{frame}


\section{Generator für TSP-Probleme}

\begin{frame}{Generator für TSP-Problem}
    \begin{itemize}
    \item Abstandsmatrix $M \in {\R}^{u^{2}}$
    \item Kosten $v,w \in [0,u]$
    \item $dst(v,w) = M[v*u+w] = M[w*u+v] \Rightarrow$ 
    doppelt so groß wie nötig
    \end{itemize}
\end{frame}


\section{Evaluation}

\begin{frame}{Evaluation}
    \begin{itemize}
        \item Intel® Core™ i5-5257U CPU @ 2.70GHz × 4, 8GB Ram
        \item n=10: 470ms 
        \item n=15: 44min running, then stopped (ca. 47h)
        \item n=20: ca. 87442560h
        \item problem is NP-hard
    \end{itemize}
\end{frame}


\begin{frame}{Evaluation Seed $\in [1,11], n=10$ }
\begin{tabular}{ l | c | c |c |c |c |c |c | c |r }
   1 & 2 & 3 & 4 & 5 & 6 & 7 & 8 & 9 & 10 & 11  \\
   440 & 466 & 435 & 467 & 444 & 448 & 439 & 426 & 485 & 421 & 437 \\
\end{tabular}
\end{frame}

\end{document}


% Local Variables:
% TeX-engine: xetex
% End:
